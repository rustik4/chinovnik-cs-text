\documentclass{article}          % загружаем стандартный LaTeX-класс. Можно также использовать класс memoir

\usepackage{polyglossia}         % для переключения языка подгружаем пакет polyglossia
\setmainlanguage{churchslavonic} % объявляем основным языком документа церковнославянский (churchslavonic)
%\usepackage{microtype}
\usepackage{churchslavonic}      % грузим пакет churchslavonic для настройки параметров печати и макрокоманд
\usepackage{lettrine}            % этот пакет нужен для буквиц
\usepackage{fwlw}                % а этот пакет позволяет получить доступ к первому слову на странице

\usepackage[a5paper]{geometry} % размер бумаги, здесь же можно задать и поля; a5 - как Часосло́въ, a6 - как карманный Слꙋже́бникъ; шрифты подогнаны под a5, для a6 нужно немного уменьшить
\makeatletter
% Определим новый стиль нумерации страниц "cuNum": в центре - номер страницы цыфирью, справа - первое слово
% следующей страницы
\def\ps@cuNum{
\def\@oddfoot{\footnotesize\centering\hfil{\Large\cuNum{\value{page}}}\hfil\hbox to 0pt{\hss\usebox\NextWordBox}}%
\let\@evenfoot\@oddfoot
}

% Зададим стиль буквиц: размер - две строки
\def\cu@lettrine{\lettrine[lines=2,nindent=0pt,loversize=-0.15]}
\def\cuLettrine{\cu@tokenizeletter\cu@lettrine}
\renewcommand{\LettrineFontHook}{\indiction\cuKinovarColor}
\makeatother

% Зададим шрифты
\newfontfamily\churchslavonicfont[Script=Cyrillic,Ligatures=TeX,HyphenChar=_]{PonomarUnicode.otf}[SmallCapsFont={PonomarUnicode.otf}] % у этого шрифта нет раздела для капители, а после индиктиона выбирается капитель
\newfontfamily\indiction{IndictionUnicode.otf}

% Макрокоманда для вывода заголовка (встроенные в LaTeX заголовки слишком отличаются от стиля ЦСЯ)
\newcommand{\Header}[1]{{\pagebreak\LARGE\centering\cuKinovar{#1}\par}\nopagebreak} % заголовок главы
\newcommand{\header}[1]{{\vspace{1em}\large\centering\cuKinovar{#1}\par}\nopagebreak} % заголовок отдельного блока
\newcommand{\notes}[1]{{\large\cuKinovar{#1}}} % указания
\newcommand{\notKinovar}[1]{\textcolor{black}{\cuKinovar #1}}

\hyphenpenalty=7000 % если увеличить этот параметр, то будет меньше переносов. По умолчанию 1000
\setlength\emergencystretch{0.75em}% разрешим растягивать пробелы между словами, чтобы избежать переносов

% немного растянем межстрочное расстояния - для лучшей читабельности
\linespread{1.2} % 1.2 значит: на 20% больше


\begin{document}
\begin{titlepage}
  \centering
  %\Huge
  \fontsize{1.3cm}{1.3cm}\selectfont
  {\cuKinovar {Чино́вникъ а҆рхїере́йскагѡ свѧщеннослꙋже́нїѧ}}
  \vfill
  \Large
  По изда́нїю Моско́вской~Патрїархі́и лѣ́та~\cuNum{1982}
  \par Каза́нь \cuNum{\year}
 \date{}
\end{titlepage}

% Задаём стиль оформления страниц "cuNum" (определённый выше)
\pagestyle{cuNum}
\Large             % переключимся на крупный шрифт, дабы удобнее читать

\Header{На вече́рни}
\vspace{1em}
\notes{Ѿкрове́нною главо́ю, глаго́летъ моли̑твы свѣти̑льнычныѧ:}

\header{Моли́тва \cuNum{1}}
\cuLettrine
Гдⷭ҇и ще́дрый и҆ ми́лостивый, долготерпѣли́ве и҆ многоми́лостиве, внꙋшѝ моли́твꙋ на́шꙋ и҆ вонмѝ гла́сꙋ моле́нїѧ на́шегѡ,
сотворѝ съ на́ми зна́менїе во бла́го: наста́ви на́съ на пꙋ́ть тво́й, є҆́же ходи́ти во и҆́стинѣ твое́й: возвеселѝ сердца̀ на̑ша,
во є҆́же боѧ́тисѧ и҆́мене твоегѡ̀ свѧта́гѡ: занѐ ве́лїй є҆сѝ ты̀ и҆ творѧ́й чꙋдеса̀, ты̀ є҆сѝ бг҃ъ є҆ди́нъ, и҆ нѣ́сть подо́бенъ тебѣ̀ въ бозѣ́хъ,
гдⷭ҇и: си́ленъ въ ми́лости и҆ бла́гъ въ крѣ́пости, во є҆́же помога́ти, и҆ оу҆тѣша́ти, и҆ спаса́ти всѧ̑ оу҆пова́ющыѧ во и҆́мѧ ст҃о́е твоѐ.
\par
\cuKinovar Ꙗ҆́кѡ подоба́етъ тебѣ̀ всѧ́каѧ сла́ва, че́сть и҆ поклоне́нїе, ѻ҆ц҃ꙋ̀ и҆ сн҃ꙋ и҆ ст҃о́мꙋ дх҃ꙋ, ны́нѣ и҆ при́снѡ и҆ во вѣ́ки вѣкѡ́въ.
\cuKinovar А҆ми́нь.

\header{Моли́тва \cuNum{2}}
\cuLettrine
Гдⷭ҇и, да не ꙗ҆́ростїю твое́ю ѡ҆бличи́ши на́съ, нижѐ гнѣ́вомъ твои́мъ нака́жеши на́съ, но сотворѝ съ на́ми по ми́лости твое́й,
врачꙋ̀ и҆ и҆сцѣли́телю дꙋ́шъ на́шихъ: наста́ви на́съ ко приста́нищꙋ хотѣ́нїѧ твоегѡ̀: просвѣтѝ ѻ҆́чи серде́цъ на́шихъ
въ позна́нїе твоеѧ̀ и҆́стины: и҆ да́рꙋй на́мъ про́чее настоѧ́щагѡ днѐ ми́рное и҆ безгрѣ́шное, и҆ всѐ вре́мѧ живота̀ на́шегѡ,
моли́твами ст҃ы́ѧ бцⷣы и҆ всѣ́хъ ст҃ы́хъ.
\par
\cuKinovar Ꙗ҆́кѡ твоѧ̀ держа́ва и҆ твоѐ є҆́сть ца́рство, и҆ си́ла, и҆ сла́ва, ѻ҆ц҃а̀ и҆ сн҃а и҆ ст҃а́гѡ дх҃а, ны́нѣ и҆ при́снѡ и҆ во вѣ́ки вѣкѡ́въ.
\cuKinovar А҆ми́нь.

\header{Моли́тва \cuNum{3}}
\cuLettrine
Гдⷭ҇и бж҃е на́шъ, помѧнѝ на́съ грѣ́шныхъ и҆ непотре́бныхъ ра̑бъ твои́хъ, внегда̀ призыва́ти на́мъ ст҃о́е покланѧ́емое и҆́мѧ твоѐ, и҆ не посрамѝ на́съ ѿ ча́ѧнїѧ ми́лости твоеѧ̀: но да́рꙋй на́мъ, гдⷭ҇и, всѧ̑ ꙗ҆̀же ко спасе́нїю прошє́нїѧ, и҆ сподо́би на́съ люби́ти и҆ боѧ́тисѧ тебѐ ѿ всегѡ̀ се́рдца на́шегѡ и҆ твори́ти во всѣ́хъ во́лю твою̀.
\par
\cuKinovar Ꙗ҆́кѡ бла́гъ и҆ человѣколю́бецъ бг҃ъ є҆сѝ, и҆ тебѣ̀ сла́вꙋ возсыла́емъ, ѻ҆ц҃ꙋ̀ и҆ сн҃ꙋ и҆ ст҃о́мꙋ дх҃ꙋ, ны́нѣ и҆ при́снѡ и҆ во вѣ́ки вѣкѡ́въ.
\cuKinovar А҆ми́нь.

\header{Моли́тва \cuNum{4}}
\cuLettrine
Немо́лчными пѣ́сньми и҆ непреста́нными славословле́ньми ѿ ст҃ы́хъ си́лъ воспѣва́емый, и҆спо́лни оу҆ста̀ на̑ша хвале́нїѧ твоегѡ̀, є҆́же пода́ти вели́чествїе и҆̀мени твоемꙋ̀ ст҃о́мꙋ: и҆ да́ждь на́мъ оу҆ча́стїе и҆ наслѣ́дїе со всѣ́ми боѧ́щимисѧ тебѐ и҆̀стиною и҆ хранѧ́щими за́повѣди твоѧ̑, моли́твами ст҃ы́ѧ бцⷣы и҆ всѣ́хъ ст҃ы́хъ твои́хъ.
\par
\cuKinovar Ꙗ҆́кѡ подоба́етъ тебѣ̀ всѧ́каѧ сла́ва, че́сть и҆ поклоне́нїе, ѻц҃ꙋ̀ и҆ сн҃ꙋ и҆ ст҃о́мꙋ дх҃ꙋ, ны́нѣ и҆ при́снѡ и҆ во вѣ́ки вѣкѡ́въ.
\cuKinovar А҆ми́нь.

\header{Моли́тва \cuNum{5}}
\cuLettrine
Гдⷭ҇и, гдⷭ҇и, пречи́стою твое́ю дла́нїю содержа́й всѧ́чєскаѧ, долготерпѧ́й на всѣ́хъ на́съ и҆ ка́ѧйсѧ ѡ҆ ѕло́бахъ на́шихъ, помѧнѝ щедрѡ́ты твоѧ̑ и҆ ми́лость твою̀, посѣти́ ны твое́ю бла́гостїю и҆ да́ждь на́мъ и҆збѣжа́ти и҆ про́чее настоѧ́щагѡ днѐ, твое́ю благода́тїю, ѿ разли́чныхъ ко́зней лꙋка́вагѡ, и҆ ненавѣ́тнꙋ жи́знь на́шꙋ соблюдѝ благода́тїю всест҃а́гѡ твоегѡ̀ дх҃а.
\par
\cuKinovar Ми́лостїю и҆ человѣколю́бїемъ є҆диноро́днагѡ твоегѡ̀ сн҃а, съ ни́мже благослове́нъ є҆сѝ, со всест҃ы́мъ и҆ благи́мъ и҆ животворѧ́щимъ твои́мъ дх҃омъ, ны́нѣ и҆ при́снѡ и҆ во вѣ́ки вѣкѡ́въ.
\cuKinovar А҆ми́нь.

\header{Моли́тва \cuNum{6}}
\cuLettrine
Бж҃е вели́кїй и҆ ди́вный, неи҆сповѣди́мою бла́гостїю и҆ бога́тымъ про́мысломъ оу҆правлѧ́ѧй всѧ́чєскаѧ и҆ мїрска̑ѧ на́мъ блага̑ѧ дарова́вый, и҆ спорꙋчи́вый на́мъ ѡ҆бѣща́нное ца́рство, дарова́нными благи́ми, пꙋтесотвори́вый на́мъ и҆ днѐ преше́дшꙋю ча́сть ѿ всѧ́кагѡ оу҆клони́тисѧ ѕла̀: да́рꙋй на́мъ и҆ про́чее непоро́чнѡ соверши́ти предъ ст҃о́ю сла́вою твое́ю, пѣ́ти тѧ̀ є҆ди́наго блага́го и҆ человѣколюби́ваго бг҃а на́шего.
\par
\cuKinovar Ꙗ҆́кѡ ты̀ є҆сѝ бг҃ъ на́шъ, и҆ тебѣ̀ сла́вꙋ возсыла́емъ, ѻц҃ꙋ̀ и҆ сн҃ꙋ и҆ ст҃о́мꙋ дх҃ꙋ, ны́нѣ и҆ при́снѡ и҆ во вѣ́ки вѣкѡ́въ.
\cuKinovar А҆ми́нь.

\header{Моли́тва \cuNum{7}}
\cuLettrine
Бж҃е вели́кїй и҆ вы́шнїй, є҆ди́нъ и҆мѣ́ѧй безсме́ртїе, во свѣ́тѣ живы́й непристꙋ́пнѣмъ, всю̀ тва́рь премꙋ́дростїю созда́вый, раздѣли́вый междꙋ̀ свѣ́томъ и҆ междꙋ̀ тьмо́ю, и҆ со́лнце положи́вый во ѻ҆́бласть днѐ, лꙋнꙋ́ же и҆ ѕвѣ́зды во ѻ҆́бласть но́щи, сподо́бивый на́съ грѣ́шныхъ и҆ въ настоѧ́щїй ча́съ предвари́ти лицѐ твоѐ и҆сповѣ́данїемъ и҆ вече́рнее тебѣ̀ славосло́вїе принестѝ: са́мъ, человѣколю́бче, и҆спра́ви моли́твꙋ на́шꙋ ꙗ҆́кѡ кади́ло предъ тобо́ю и҆ прїимѝ ю҆̀ въ воню̀ благоꙋха́нїѧ: пода́ждь же на́мъ настоѧ́щїй ве́черъ и҆ приходѧ́щꙋю но́щь ми́рнꙋ, ѡ҆блецы́ ны во ѻ҆рꙋ́жїе свѣ́та, и҆зба́ви ны̀ ѿ стра́ха нощна́гѡ и҆ всѧ́кїѧ ве́щи, во тьмѣ̀ преходѧ́щїѧ, и҆ да́ждь со́нъ, є҆го́же во оу҆покое́нїе не́мощи на́шей дарова́лъ є҆сѝ, всѧ́кагѡ мечта́нїѧ дїа́волѧ ѿчꙋжде́нный. Є҆́й, влⷣко, благи́хъ пода́телю, да и҆ на ло́жахъ на́шихъ оу҆милѧ́ющесѧ, помина́емъ въ нощѝ и҆́мѧ твоѐ и҆, поꙋче́нїемъ твои́хъ за́повѣдей просвѣща́еми, въ ра́дости дꙋше́внѣй воста́немъ ко славосло́вїю твоеѧ̀ бла́гости, молє́нїѧ и҆ моли̑твы твоемꙋ̀ благоꙋтро́бїю приносѧ́ще ѡ҆ свои́хъ согрѣше́нїихъ и҆ всѣ́хъ люде́й твои́хъ, ꙗ҆̀же, моли́твами ст҃ы́ѧ бцⷣы, ми́лостїю посѣтѝ.
\par
\cuKinovar Ꙗ҆́кѡ бла́гъ и҆ человѣколю́бецъ бг҃ъ є҆сѝ, и҆ тебѣ̀ сла́вꙋ возсыла́емъ, ѻ҆ц҃ꙋ̀ и҆ сн҃ꙋ и҆ ст҃о́мꙋ дх҃ꙋ, ны́нѣ и҆ при́снѡ и҆ во вѣ́ки вѣкѡ́въ.
\cuKinovar А҆ми́нь.

\notes{И҆спо́лньшꙋсѧ же предначина́тельномꙋ ѱалмꙋ́, глаго́летъ сщ҃е́нникъ и҆лѝ дїа́конъ, а҆́ще є҆́сть, и҆зше́дъ сѣ́верною страно́ю и҆ ста́въ на ѡ҆бы́чнѣмъ мѣ́стѣ а҆мвѡ́на, є҆ктенїю̀ сїю̀: \notKinovar {Ми́ромъ гдⷭ҇ꙋ помо́лимсѧ.}}

\header{Моли́тва вхо́да}
\cuLettrine
Ве́черъ, и҆ заꙋ́тра, и҆ полꙋ́дне, хва́лимъ, благослови́мъ, благодари́мъ и҆ мо́лимсѧ тебѣ̀, влⷣко всѣ́хъ: и҆спра́ви моли́твꙋ на́шꙋ, ꙗ҆́кѡ кади́ло предъ тобо́ю, и҆ не оу҆клонѝ серде́цъ на́шихъ въ словеса̀ и҆лѝ въ помышлє́нїѧ лꙋка́вствїѧ: но и҆зба́ви на́съ ѿ всѣ́хъ ловѧ́щихъ дꙋ́шы на́шѧ, ꙗ҆́кѡ къ тебѣ̀, гдⷭ҇и, гдⷭ҇и, ѻ҆́чи на́ши, и҆ на тѧ̀ оу҆пова́хомъ, да не посрами́ши на́съ, бж҃е на́шъ.
\par
\cuKinovar Ꙗ҆́кѡ подоба́етъ тебѣ̀ всѧ́каѧ сла́ва, че́сть и҆ поклоне́нїе, ѻ҆ц҃ꙋ̀ и҆ сн҃ꙋ и҆ ст҃о́мꙋ дх҃ꙋ, ны́нѣ и҆ при́снѡ и҆ во вѣ́ки вѣкѡ́въ.
\cuKinovar А҆ми́нь.

\header{Моли́тва главоприклоне́нїѧ}
\cuLettrine
Гдⷭ҇и бж҃е на́шъ, приклони́вый небеса̀ и҆ соше́дый на спасе́нїе ро́да человѣ́ческагѡ, при́зри на рабы̑ твоѧ̑ и҆ на достоѧ́нїе твоѐ: тебѣ́ бо стра́шномꙋ и҆ человѣколю́бцꙋ сꙋдїѝ твоѝ рабѝ подклони́ша главы̑, своѧ̑ же покори́ша вы̑ѧ, не ѿ человѣ̑къ ѡ҆жида́юще по́мощи, но твоеѧ̀ ѡ҆жида́юще ми́лости и҆ твоегѡ̀ ча́юще спасе́нїѧ, ꙗ҆̀же сохранѝ на всѧ́кое вре́мѧ, и҆ по настоѧ́щемъ ве́черѣ, и҆ въ приходѧ́щꙋю но́щь, ѿ всѧ́кагѡ врага̀, ѿ всѧ́кагѡ проти́внагѡ дѣ́йства дїа́вольскагѡ, и҆ ѿ помышле́нїй сꙋ́етныхъ, и҆ воспомина́нїй лꙋка́выхъ.

\Header{На лїті́и}
\notes{По сконча́нїи моли́твъ, глаго́лемыхъ дїа́кономъ, а҆рхїере́й возгла́съ:}
\cuLettrine
Ꙋ҆слы́ши ны̀, бж҃е, спаси́телю на́шъ, оу҆пова́нїе всѣ́хъ концє́въ землѝ и҆ сꙋ́щихъ въ мо́ри дале́че, и҆ ми́лостивъ, ми́лостивъ бꙋ́ди, влⷣко, ѡ҆ грѣсѣ́хъ на́шихъ и҆ поми́лꙋй ны̀. Ми́лостивъ бо и҆ человѣколю́бецъ бг҃ъ є҆сѝ, и҆ тебѣ̀ сла́вꙋ возсыла́емъ, ѻ҆ц҃ꙋ̀ и҆ сн҃ꙋ и҆ ст҃о́мꙋ дх҃ꙋ, ны́нѣ и҆ при́снѡ и҆ во вѣ́ки вѣкѡ́въ.
\par \notes{Ли́къ: \notKinovar{А҆ми́нь.}}
\par \notes {Та́же а҆рхїере́й:} \cuKinovar Ми́ръ всѣ̑мъ.
\par \notes{Ли́къ: \notKinovar{И҆ дꙋ́хови твоемꙋ̀.}}
\par \notes{Дїа́конъ: \notKinovar{Главы̑ на́шѧ гдⷭ҇еви прикло́нимъ.} И҆ всѣ̑мъ прикло́ньшымъ главы̑, мо́литсѧ а҆рхїере́й велегла́снѡ:}
\cuLettrine
Влⷣко многоми́лостиве, гдⷭ҇и і҆и҃се хрⷭ҇тѐ бж҃е на́шъ,
моли́твами всепречи́стыѧ влⷣчцы на́шеѧ бцⷣы и҆ приснодѣ́вы марі́и,
си́лою честна́гѡ и҆ животворѧ́щагѡ креста̀,
предста́тельствы честны́хъ небе́сныхъ си́лъ безпло́тныхъ,
честна́гѡ сла́внагѡ проро́ка, предте́чи и҆ крести́телѧ і҆ѡа́нна,
ст҃ы́хъ сла́вныхъ и҆ всехва́льныхъ а҆по́стѡлъ,
ст҃ы́хъ сла́вныхъ и҆ добропобѣ́дныхъ мꙋ́ченикѡвъ,
преподо́бныхъ и҆ бг҃оно́сныхъ ѻ҆тє́цъ на́шихъ,
и҆̀же во ст҃ы́хъ ѻ҆тє́цъ на́шихъ и҆ вселе́нскихъ вели́кихъ оу҆чи́телей и҆ ст҃и́телей васі́лїѧ вели́кагѡ, григо́рїѧ бг҃осло́ва и҆ і҆ѡа́нна златоꙋ́стагѡ,
и҆́же во ст҃ы́хъ ѻ҆тца̀ на́шегѡ нїкола́ѧ, а҆рхїепі́скопа мѵ́ръ лѷкі́йскихъ, чꙋдотво́рца,
ст҃ы́хъ равноапо́стольныхъ меѳо́дїѧ и҆ кѷрі́лла, оу҆чи́телей слове́нскихъ,
свѧта́гѡ равноапо́стольнагѡ вели́кагѡ кнѧ́зѧ влади́мїра
и҆ свѧты́ѧ равноапо́стольныѧ вели́кїѧ кнѧги́ни рѡссі́йскїѧ ѻ҆́льги,
и҆̀же во ст҃ы́хъ ѻ҆те́цъ на́ших̾, всеѧ̀ рѡссі́и чꙋдотво́рцєвъ мїхаи́ла, петра̀, а҆леѯі́ѧ, і҆ѡ́ны, фїлі́ппа, є҆рмоге́на, і҆нноке́нтїѧ, мака́рїѧ, дими́трїѧ, митрофа́на, тѵ́хѡна, ѳеодо́сїѧ, і҆ѡаса́фа, пїтѵрі́ма, і҆ѡа́нна, і҆нноке́нтїѧ и҆ сѡфро́нїѧ,
ст҃ы́хъ сла́вныхъ и҆ добропобѣ́дныхъ мꙋ́ченикѡвъ:
ст҃а́гѡ сла́внагѡ великомч҃нка, побѣдоно́сца и҆ чꙋдотво́рца геѡ́ргїѧ,
ст҃а́гѡ великомч҃нка и҆ цѣли́телѧ пантелеи́мѡна,
ст҃ы́ѧ великомч҃нцы варва́ры
и҆ ст҃ы́хъ благовѣ́рныхъ рѡссі́йскихъ кнѧзе́й и҆ страстоте́рпцєвъ бори́са, глѣ́ба и҆ и҆́горѧ,
преподо́бныхъ и҆ бг҃оно́сныхъ ѻ҆тє́цъ на́ших̾ а҆нтѡ́нїѧ и҆ ѳеодо́сїѧ и҆ про́чихъ чꙋдотво́рцєв̾ пече́рскихъ,
и҆ преподо́бныхъ и҆ бг҃оно́сныхъ ѻ҆тє́цъ на́шихъ
се́ргїѧ, и҆гꙋ́мена ра́донежскагѡ, чꙋдотво́рца, и҆ серафі́ма саро́вскагѡ, чꙋдотво́рца,
и҆ преподо́бнагѡ и҆ бг҃оно́снагѡ ѻ҆тца̀ на́шегѡ і҆́ѡва, и҆гꙋ́мена поча́евскагѡ, чꙋдотво́рца,
и҆ ст҃а́гѡ, \cuKinovar{и҆́мⷬ҇къ,} \notes{є҆гѡ́же є҆́сть хра́мъ и҆ є҆гѡ́же є҆́сть де́нь,}
ст҃ы́хъ и҆ пра́ведныхъ бг҃оѻтє́цъ і҆ѡакі́ма и҆ а҆́нны и҆ всѣ́хъ ст҃ы́хъ твои́хъ,
благопрїѧ́тнꙋ сотворѝ моли́твꙋ на́шꙋ, да́рꙋй на́мъ ѡ҆ставле́нїе прегрѣше́нїй на́шихъ, покры́й на́съ кро́вомъ крилꙋ̑ твоє́ю, ѿженѝ ѿ на́съ всѧ́каго врага̀ и҆ сꙋпоста́та, оу҆мирѝ на́шꙋ жи́знь, гдⷭ҇и, поми́лꙋй на́съ и҆ мі́ръ тво́й и҆ спасѝ дꙋ́шы на́шѧ, ꙗ҆́кѡ бла́гъ и҆ человѣколю́бецъ.

\notes{Та́же начина́емъ стїхи̑ры стїхѡ́вны, и҆ пою́ще вхо́димъ во хра́мъ.
\notKinovar{Сла́ва, и҆ ны́нѣ:} Бг҃оро́диченъ. Та́же: \notKinovar{Ны́нѣ ѿпꙋща́еши:} Трист҃о́е. По \notKinovar{Ѻ҆́ч҃е на́шъ:}}
\par \notes{Возглаше́нїе:}
\par
\cuKinovar Ꙗ҆́кѡ твоѐ є҆́сть ца́рство, и҆ си́ла, и҆ сла́ва, ѻ҆ц҃а̀ и҆ сн҃а и҆ ст҃а́гѡ дх҃а, ны́нѣ и҆ при́снѡ и҆ во вѣ́ки вѣкѡ́въ.
\par \notes{Ли́къ: \notKinovar{А҆ми́нь.}}
\notes{И҆ мы̀ тропа́рь пра́здника глаго́лемъ три́жды. Предлага́ютсѧ же на оу҆гото́ваннѣмъ столѣ̀ ра́ди бл҃гослове́нїѧ пѧ́ть хлѣ́бѡвъ, пшени́ца, и҆ два̀ сосꙋ́да на сїѐ оу҆стро́єнныѧ, є҆ди́нъ и҆спо́лненъ вїна̀ ѿ плода̀ ло́знагѡ, дрꙋгі́й же є҆ле́а. Дїа́конъ же кади́тъ ѡ҆́крестъ стола̀, и҆ а҆рхїере́а. А҆рхїере́й же, взе́мъ є҆ди́нъ хлѣ́бъ, зна́менꙋетъ и҆́мъ про́чыѧ хлѣ́бы и҆ глаго́летъ мл҃твꙋ сїю̀ велегла́снѡ: Є҆гда́ же глаго́летъ: \notKinovar{Са́мъ бл҃гословѝ:} тогда̀ десни́цею оу҆казꙋ́етъ на предложє́нныѧ хлѣ́бы, пшени́цꙋ, вїно̀ и҆ є҆ле́й.}
\cuLettrine
Гдⷭ҇и і҆и҃се хрⷭ҇тѐ бж҃е на́шъ, благослови́вый пѧ́ть хлѣ́бѡвъ и҆ пѧ́ть ты́сѧщъ насы́тивый, са́мъ благословѝ и҆ хлѣ́бы сїѧ̑, пшени́цꙋ, вїно̀ и҆ є҆ле́й, и҆ оу҆мно́жи сїѧ̑ во гра́дѣ се́мъ \cuKinovar{[и҆лѝ:} въ ве́си се́й, \cuKinovar{и҆лѝ:} во ст҃ѣ́й ѻ҆би́тели се́й\cuKinovar{]} и҆ во все́мъ мі́рѣ твое́мъ, и҆ вкꙋша́ющыѧ ѿ ни́хъ вѣ̑рныѧ ѡ҆свѧтѝ.
Ꙗ҆́кѡ ты̀ є҆сѝ благословлѧ́ѧй и҆ ѡ҆свѧща́ѧй всѧ́чєскаѧ, хрⷭ҇тѐ бж҃е на́шъ, и҆ тебѣ̀ сла́вꙋ возсыла́емъ, со безнача́льнымъ твои́мъ ѻ҆ц҃е́мъ и҆ всест҃ы́мъ и҆ благи́мъ и҆ животворѧ́щимъ твои́мъ дх҃омъ, ны́нѣ и҆ при́снѡ и҆ во вѣ́ки вѣкѡ́въ.
\par \notes{Ли́къ: \notKinovar{А҆ми́нь.} И҆ по а҆ми́нѣ, а҆́бїе:}
\par \notes{\notKinovar{Бꙋ́ди и҆́мѧ гдⷭ҇не бл҃гослове́но ѿны́нѣ и҆ до вѣ́ка.} Три́жды.}
\par \notes{И҆ глаго́лемъ: \notKinovar{Бл҃гословлю̀ гдⷭ҇а на всѧ́кое вре́мѧ:} да́же до: \notKinovar{Не лиша́тсѧ всѧ́кагѡ бл҃га.} А҆рхїере́й же ѿше́дъ стои́тъ предъ ст҃ы́ми две́рьми.}
\par \notes{По и҆сполне́нїи же ѱалма̀, а҆рхїере́й къ наро́дꙋ глаго́летъ:}
\cuLettrine
Благослове́нїе гдⷭ҇не на ва́съ, тогѡ̀ благода́тїю и҆ человѣколю́бїемъ, всегда̀, ны́нѣ и҆ при́снѡ, и҆ во вѣ́ки вѣкѡ́въ.
\par \notes{Ли́къ: \notKinovar{А҆ми́нь.}}
\par \notes{И҆ а҆́бїе полага́етсѧ чте́нїе. \notKinovar{Сла́ва въ вы́шнихъ бг҃ꙋ,} три́жды, и҆ \notKinovar{Гдⷭ҇и, оу҆стнѣ̀ моѝ ѿве́рзеши,} два́жды, и шестоѱа́лмїе.}


\end{document}

