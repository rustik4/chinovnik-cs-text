\documentclass{article}          % загружаем стандартный LaTeX-класс. Можно также использовать класс memoir

\usepackage{polyglossia}         % для переключения языка подгружаем пакет polyglossia
\setmainlanguage{churchslavonic} % объявляем основным языком документа церковнославянский (churchslavonic)
\usepackage{microtype}
\usepackage{churchslavonic}      % грузим пакет churchslavonic для настройки параметров печати и макрокоманд
\usepackage{lettrine}            % этот пакет нужен для буквиц
\usepackage{fwlw}                % а этот пакет позволяет получить доступ к первому слову на странице

\makeatletter
% Определим новый стиль нумерации страниц "cuNum": в центре - номер страницы цыфирью, справа - первое слово
% следующей страницы
\def\ps@cuNum{
\def\@oddfoot{\footnotesize\centering\hfil{\Large\cuNum{\value{page}}}\hfil\hbox to 0pt{\hss\usebox\NextWordBox}}%
\let\@evenfoot\@oddfoot
}

% Зададим стиль буквиц: размер - три строки
\def\cu@lettrine{\lettrine[lines=2,findent=0pt,nindent=0pt]}
\def\cuLettrine{\cu@tokenizeletter\cu@lettrine}
\renewcommand{\LettrineFontHook}{\indiction\cuKinovarColor}
\makeatother

% Зададим шрифты
\newfontfamily\churchslavonicfont[Script=Cyrillic,Ligatures=TeX,HyphenChar=_]{PonomarUnicode.otf}
\newfontfamily\indiction{IndictionUnicode.otf}

% Макрокоманда для вывода заголовка (встроенные в LaTeX заголовки слишком отличаются от стиля ЦСЯ)
\newcommand{\header}[1]{{\LARGE\centering\cuKinovar{#1}\par}}

%
\hyphenpenalty=7000 % если увеличить этот параметр, то будет меньше переносов. По умолчанию 1000
\setlength\emergencystretch{0.75em}% разрешим растягивать пробелы между словами, чтобы избежать переносов

% немного растянем межстрочное расстояния - для лучшей читабельности
\linespread{1.2} % 1.2 значит: на 20% больше


\begin{document}
\begin{titlepage}
  \centering
  \Huge
  {\cuKinovar {Чино́вникъ а҆рхїере́йскагѡ свѧщеннослꙋже́нїѧ}}
  \vfill
  \Large
  На́брано съ изда́нїѧ Моско́вской~Патрїа́рхїи лѣ́та~\cuNum{1982} во гра́де Каза́ни лѣ́та~\cuNum{2018}
 \date{}
\end{titlepage}

% Задаём стиль оформления страниц "cuNum" (определённый выше)
\pagestyle{cuNum}
\Large             % переключимся на крупный шрифт, дабы удобнее читать

\header{На вече́рни}
\vspace{2em}
\cuKinovar {Ѿкрове́нною главо́ю, глаго́летъ моли̑твы свети̑льнычныѧ:}
\vspace{1em}

\begin{center}
\cuKinovar {Моли́тва \cuNum{1}}
\end{center}
\cuLettrine
Гдⷭ҇и ще́дрый и҆ ми́лостивый, долготерпѣли́ве и҆ многоми́лостиве, внꙋшѝ моли́твꙋ на́шꙋ и҆ вонмѝ гла́сꙋ моле́нїѧ на́шего,
сотворѝ съ на́ми зна́менїе во бла́го: наста́ви на́съ на пꙋ́ть тво́й, є҆́же ходи́ти во и҆́стинѣ твое́й: возвеселѝ сердца̀ на̑ша,
во є҆́же боѧ́тисѧ и҆́мене твоегѡ̀ свѧта́гѡ: занѐ ве́лїй є҆сѝ ты̀ и҆ творѧ́й чꙋдеса̀, ты̀ є҆сѝ бг҃ъ є҆ди́нъ, и҆ нѣ́сть подо́бенъ тебѣ̀ въ бозѣ́хъ,
гдⷭ҇и: си́ленъ въ ми́лости и҆ бла́гъ въ крѣ́пости, во є҆́же помога́ти, и҆ оу҆тѣша́ти, и҆ спаса́ти всѧ̑ оу҆пова́ющыѧ во и҆́мѧ ст҃о́е твоѐ.
\par
\cuKinovar Ꙗ҆́кѡ подоба́етъ тебѣ́ всѧ́каѧ сла́ва, че́сть и҆ поклоне́нїе, ѻ҆ц҃ꙋ̀ и҆ сн҃ꙋ и҆ ст҃о́мꙋ дх҃ꙋ, ны́нѣ и҆ при́снѡ и҆ во вѣ́ки вѣкѡ́въ.
\cuKinovar А҆ми́нь.

\begin{center}
\cuKinovar {Моли́тва \cuNum{2}}
\end{center}
\cuLettrine
Гдⷭ҇и, да не ꙗ҆́ростїю твое́ю ѡ҆бличи́ши на́съ, нижѐ гнѣ́вомъ твои́мъ нака́жеши на́съ, но сотворѝ съ на́ми по ми́лости твое́й,
врачꙋ̀ и҆ и҆сцѣли́телю дꙋ́шъ на́шихъ: наста́ви на́съ ко приста́нищꙋ хотѣ́нїѧ твоегѡ̀: просвѣтѝ ѻ҆́чи серде́цъ на́шихъ
въ позна́нїе твоеѧ̀ и҆́стины: и҆ да́рꙋй на́мъ про́чее настоѧ́щагѡ днѐ ми́рное и҆ безгрѣ́шное, и҆ всѐ вре́мѧ живота̀ на́шегѡ,
моли́твами ст҃ы́ѧ бцⷣы и҆ все́хъ ст҃ы́хъ.
\par
\cuKinovar Ꙗ҆́кѡ твоѧ̀ держа́ва и҆ твоѐ є҆́сть ца́рство, и҆ си́ла, и҆ сла́ва, ѻ҆ц҃а̀ и҆ сн҃а и҆ ст҃агѡ дх҃а, ны́нѣ и҆ при́снѡ и҆ во вѣ́ки вѣкѡ́въ.
\cuKinovar А҆ми́нь.

\end{document}
